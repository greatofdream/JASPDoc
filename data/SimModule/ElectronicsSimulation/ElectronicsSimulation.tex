\graphicspath{{data/SimModule/ElectronicsSimualtion/}}

\chapter[Electronics Simulation]
{Electronics Simulation}
\section{Usage}
\texttt{JPSim} runs Electronics Simulation default. Electronics Simulation can be shut down using parameters \texttt{-e ElectronicsSwitch}.
\begin{lstlisting}[language=bash]
$ JPSim [-e (OFF/ON)] [-dn nDarkNoise[0/1/2/n]]
\end{lstlisting}
\begin{itemize}
    \item {\tt ElectronicsSwitch} or \texttt{-e} and this parameter needs be set mandatory.
    \item {\tt nDarkNoise} default is 1(defined in JPSimTrigger.cc). PreTriggerNoDarkNoise,PreTrigger, PreTriggerFullDarkNoiseFast, PreTriggerFullDarkNoise
    \item 
\end{itemize}
\section{ElecModule Code Structure}
Code files of Electronics Simulation is stored in \texttt{\$JASP\_install/Simulation/ElecModule}. The Structure of above directory is shown in Figure~\ref{fig:elecstructure}.
\begin{figure}[htbp]
    \dirtree{%
    .1 ElecModule/.
    .2 include/.
    .3 JPSimDarkNoise.hh.
    .3 JPSimElecTrigSim.hh.
    .3 JPSimElectronics.hh.
    .3 JPSimPMT.hh.
    .3 JPSimPMTManager.hh.
    .3 JPSimPreTrigger.hh.
    .3 JPSimTrigger.hh.
    .3 VDarkNoise.hh.
    .3 VElecTrigSim.hh.
    .3 VElectronics.hh.
    .3 VPreTrigger.hh.
    .3 VTrigger.hh.
    .2 src/.
    .3 JPSimDarkNoise.cc.
    .3 JPSimElectronics.cc.
    .3 JPSimPMT.cc.
    .3 JPSimPMTManager.cc.
    .3 JPSimPreTrigger.cc.
    .3 JPSimTrigger.cc.
    }
    \caption{{\tt ElectronicsSimulation} directory structure}
    \label{fig:elecstructure}
\end{figure}
\subsection{Brief Introduction to important files}
\begin{description}
    \item[JPSimPMT<.hh/.cc>] Class \texttt{JPSimPMT} set and return parameters of PMTs, 
    which will be set when JSAP initialize Detector geometry implemented by code in \texttt{JPSimDetectorConstruction.cc}.
    \texttt{JPSimPMT} contains many parameters, such as the position, electronics property. It use a charge model to describe
    the gain process of electrons. The charge model obey distribution of Poisson.
    % charge model
    \[P(0, \mu)*w*\alpha*e^{-\alpha x}\]
    The noise model is gauss distribution$N(0, rms)$.
    \item[VElecTrigSim.hh] Virtual Class \texttt{VElectrigSim}
    \item[JPSimDarkNoise<.hh/.cc>]  Class \texttt{DarkNoise}
    \item[JPSimElecTrigSim.hh]\label{JPSimElecTrigSim} \texttt{JPSimElecTrigSim.hh} defines template class \texttt{JPSimElecTrigSim} inherited from class \texttt{VElectrigSim}.
    \begin{lstlisting}[language=C++]
    template<class PreTrigger, class DarkNoise,
             class Electronics, class Trigger>
    class JPSimElecTrigSim : public VElecTrigSim
    \end{lstlisting}
    Class \texttt{JPSimElecTrigSim} is used as the combination of the above four classes and storage the objects of four classes in each event simulation. In code realization, class JPSimElecTrigSim
    is construct in construction of event simulation in file \texttt{JPSimEventAction.cc}. 
    \begin{lstlisting}[language=C++]
    // JPSimEventAction.cc: JPSimEventAction::JPSimEventAction
    ets = new JPSimElecTrigSim<JPSimPreTrigger, JPSimDarkNoise,
     JPSimElectronics, JPSimTrigger>();
    \end{lstlisting}
    \item[JPSimPreTrigger<.hh/.cc>] Class \texttt{JPSimPreTrigger} return \texttt{PreTriggers} using information of global PETime(use the computer time as start of time), not the simulated waveform.
    PreTrigger operation accurates the speed of simulation. There are 4 method can be choosed
    in JPSimPreTrigger, which are described following:
    \begin{enumerate}
        \item \texttt{PreTriggerNoDarkNoise()} Only find the intervals with hits, no dark noise added.
        \item \texttt{PreTrigger()} Only find the intervals with hits, add dark noise and pass them into hitListIntvl
        \item \texttt{PreTriggerFullDarkNoiseFast()} Generate dark noise of the whole segment to pre-trigger the dark noise accidental trigger. Very slow.
        \item \texttt{PreTriggerFullDarkNoise()}
    \end{enumerate}
    %segmentBegin and segmentEnd:
    %deltat0+detlat1<0 jisimhit<

    Dark Noise Rate is also a parameter of pmt. The Dark Noise distribution obeys Poisson distribution.

    Use nTriggerCondition to check the pmt number in interval is whether more than nTriggerCondition.
    \item[JPSimElectronics<.hh/.cc>] Class \texttt{JPSimElectronics} generates waveform according to the parameters of pmt, including charge model, basline noise, DC offset and single photon response(SPE).
    \[waveform=charge\otimes spe+rms+DCoffset\]
    After generate above waveform, \texttt{JPSimElectronics} simulate the process of ADC which change the volatage to ADC address.
    
    \item[JPSimTrigger<.hh/.cc>] Class \texttt{JPSimTrigger} gives formal triggers used to extract readout waveforms.
    \texttt{JPSimTrigger<.hh/.cc>} simulates the process of pulse amplitude discriminator to get logic signal. 
    When the waveform(negative signal) less than the threshold, generate a rectangle logic signal with
    a time width of CoincidenceWindow(default value is 50ns). If the sum of logic signal exceeds nTriggerCondition, a trigger will be
    recorded. After all the triggers generated, readout and triggerinfo will be output.
\end{description}

\subsection{related Classes or code files}
\begin{description}
    \item[G4Module/src/JPSimEventAction.cc] Class \texttt{JPSimEventAction} defines the start of Electronics Simulation, PreTrigger in function \texttt{EndOfEventAction}.
    This class is the entrance of the electronics simlation. The initialization, processing and storage of Electronics Simulation are all done in this part.

    In above section~\ref{JPSimElecTrigSim}, the initialization code of \texttt{JPSimElecTrigSim} is illustrated. The entrance of Electronics Simulation is implement in function \texttt{JPSimEventAction::EndOfEventAction}.
    This function save TruthInfo first, which is the result of detector simulation. After the storage process of TruthInfo finished, Electronics Simulation starts and save TriggerInfo.
    \begin{lstlisting}[language=C++]
        // JPSimEventAction.cc:L394
        ets->GetPreTriggerPtr()->PreTriggerNoDarkNoise(
            JPSimPMTManager::GetInstance()->HitListMap, 
            JPSimPMTManager::GetInstance()->TempHitBuffer, 
            hitListIntvl, nEventNo);
    \end{lstlisting}
    \texttt{JPSimEventAction} also control the whole process of Electronic Simulation, include PreTrigger, Electronics, Trigger.
    
    \begin{lstlisting}[language=C++]
        // JPSimEventAction.cc:L422
        ets->GetElectronicsPtr()->Shape(hitItem.first, 
                hitItem.second, WaveformList, HitListMap);
        // JPSimEventAction.cc:L428
        ets->GetTriggerPtr()->Trigger(hitItem.first, HitListMap, 
                WaveformList, TriggerInfoList, fTrackContainer, 
                TrackMap, fTruthList);
    \end{lstlisting}
    \item[G4Module/src/JPSimTrackingAction.cc] Class \texttt{JPSimTrackingAction} simulates the transportation of electrons in PMT.
    It get the time of each photon hit on the Cathode(HitTime). Then it add the TTS according to different PMTs.
    \item[DataIOModule/src/JPSimDataOutput.cc] This file defined the format of output tree, such as readout and triggerinfo.
    \item[DataIOModule/src/JPSimGDMLReader.cc] Class \texttt{JPSimGDMLReader} is called by Class \texttt{JPSimPMT}, and return property
    of PMT read from GDML files.
    \item[DetectorStructure/] Storage detector structure, include PMT positon and type.
    \item[PMTCalib/] Storage the property of different types of PMT.
\end{description}
\section{Data Flow}
\subsection{PMT property}
GDML files are read by GDML Reader. It parse GDML into different value and set properties of PMT.
\begin{lstlisting}[language=c++]
    // JPSimDetectorConstruction.cc:L
    JPSimTrigger::fCoincidenceWindow = fParser->GetConstant("Coincidence_Window")*ns;
    JPSimTrigger::fThreshold = fParser->GetConstant("Threshold");
    JPSimTrigger::nTriggerCondition = fParser->GetConstant("Trigger_Condition");
    JPSimTrigger::fTimeWindow = fParser->GetConstant("Time_Window")*ns;
    JPSimTrigger::fTriggerPosition = fParser->GetConstant("Trigger_Position")*ns;
    JPSimTrigger::fDynamicRange = fParser->GetConstant("ADC_DynamicRange");
    JPSimTrigger::nBit = fParser->GetConstant("ADC_Bit");
    JPSimTrigger::fDCOffset = fParser->GetConstant("ADC_DCOffset");
    JPSimTrigger::nBits = TMath::Power(2, JPSimTrigger::nBit);
    JPSimTrigger::nThreshold = (UInt_t)TMath::Abs((JPSimTrigger::fThreshold+JPSimTrigger::fDCOffset*1000+1000*JPSimTrigger::fDynamicRange)/(JPSimTrigger::fDynamicRange*1000/JPSimTrigger::nBits));
\end{lstlisting}
\subsection{Electronics Simulation in Detector Simulation}
\subsection{PreTrigger}
\subsection{Electronics}
\subsubsection{charge model}
\subsection{Trigger}

\section{Appendix}